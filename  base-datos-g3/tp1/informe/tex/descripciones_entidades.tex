\begin{itemize}
\item \textbf{Legislador: } Representa a una persona del poder legislativo, del cual provienen las leyes. Figura principal del problema que se clasifica en dos entidades importantes: Diputado y Senador. Los cuales tienen caracter\'isticas, requisitos y funciones distintas dentro del \lq parlamento\rq . 

\quad 

\item \textbf{C\'amara: } Representa a una c\'amara que nuclea a los legisladores seg\'un sean Diputados o Senadores. Por eso mismo, se clasifica en C\'amara de Senadores y C\'amara de Diputados. Las C\'amaras sesionan para organizarse, debatir y legislar (generando y aprobando o no proyectos de ley).

\quad

\item \textbf{Proyecto de Ley: } Representa a una propuesta de ley presentada ante el \'organo legislativo. El atributo ``Estado votaciones'' inicialmente figura como ``Iniciadas'' (I). Esto quiere decir que se puede votar el proyecto en la camara originaria. Si no se obtienen los votos necesarios para pasar a la proxima camara, el estado pasara a ``Finalizadas'' (F), de no ser pasara a ``Media Sancion'' (M). Una ves terminadas la votaciones el estado pasara a ``Finalizadas''.

\quad

\item \textbf{Ley: } Es la norma jur\'idica dictada por el legislador. Cuando un proyecto de ley figura con estado ``Finalizado'' (F) y este tuvo las votaciones necesarias para ser aprobado, se genera una nueva Ley en base al proyecto aprobado. Todas las Leyes tienen un proyecto originario, pero no todos los proyectos son transformados en ley.

\quad

\item \textbf{Voto: } Representa el voto emitido por un legislador. Tiene distintas clasificaciones como el modo y el valor del voto.

\quad

\item \textbf{Bloque Pol\'itico: } Cada corriente pol\'itica organiza a sus legisladores partidarios dentro de bloques.

\quad

\item \textbf{Partido Pol\'itico: } Es una entidad democr\'atica de inter\'es p\'ublico que nuclea seg\'un intereses e ideolog\'ias. 

\quad

\item \textbf{Bien Econ\'omico: } Representa a un objeto con valor econ\'omico

\quad

\item \textbf{Sesi\'on: } Representa a las reuniones oficiales de los legisladores en las c\'amaras. Hay 3 tipos, \textit{Preparatoria, Ordinaria,} y de \textit{Prorroga}. Las de \textit{Prorroga} extienden el periodo legislativo a las \textit{Ordinarias} en una cierta cantidad de dias. No todas las sesiones \textit{Ordinarias} son extendidas por una de \textit{Prorroga}

\quad

\item \textbf{Per\'iodo: } Representa a un intervalo de tiempo, con un principio y fin establecidos. Los legisladores obtienen su cargo por un per\'iodo determinado.

\quad

\item \textbf{Comisi\'on: } Representa a grupos de diputados, que estudian proyectos entre otras cosas.

\quad

\item \textbf{Control de Calidad: } Representa al programa de seguimiento de las votaciones de los senadores descripto en el enunciado del problema.


\end{itemize}