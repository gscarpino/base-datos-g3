\begin{itemize}
\item \textbf{Legislador: } Representa a una persona del poder legislativo, del cual provienen las leyes. Figura principal del problema que se clasifica en dos entidades importantes: Diputado y Senador. Los cuales tienen caracter\'isticas, requisitos y funciones distintas dentro del \lq parlamento\rq . 

\quad 

\item \textbf{C\'amara: } Representa a una c\'amara que nuclea a los legisladores seg\'un sean Diputados o Senadores. Por eso mismo, se clasifica en C\'amara de Senadores y C\'amara de Diputados. Las C\'amaras sesionan para organizarse, debatir y legislar (generando y aprobando o no proyectos de ley).

\quad

\item \textbf{Proyecto de Ley: } Representa a una propuesta de ley presentada ante el \'organo legislativo.

\quad

\item \textbf{Ley: } Es la norma jur\'idica dictada por el legislador, es decir, un precepto establecido por la autoridad competente.

\quad

\item \textbf{Voto: } Representa el voto emitido por un legislador. Tiene distintas clasificaciones como el modo y el valor del voto.

\quad

\item \textbf{Bloque Pol\'itico: } Cada corriente pol\'itica organiza a sus legisladores partidarios dentro de bloques.

\quad

\item \textbf{Partido Pol\'itico: } Es una entidad democr\'atica de inter\'es p\'ublico que nuclea seg\'un intereses e ideolog\'ias. 

\quad

\item \textbf{Bien Econ\'omico: } Representa a un objeto con valor econ\'omico

\quad

\item \textbf{Sesi\'on: } Representa a las reuniones oficiales de los legisladores en las c\'amaras.

\quad

\item \textbf{Per\'iodo: } Representa a un intervalo de tiempo, con un principio y fin establecidos. Los legisladores obtienen el su cargo por un per\'iodo determinado.

\quad

\item \textbf{Comisi\'on: } Representa a grupos de diputados, que estudian proyectos entre otras cosas.

\quad

\item \textbf{Control de Calidad: } Representa al programa de seguimiento de las votaciones de los senadores descripto en el enunciado del problema.


\end{itemize}