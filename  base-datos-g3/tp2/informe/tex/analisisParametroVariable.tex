\quad Al igual que en el an\'alisis previo, realizamos mediciones de los estimadores variando el par\'ametro de entrada sobre las mismas poblaciones mencionadas.\\

\quad El par\'ametro que se var\'ia en todos los estimadores representa la cantidad de \textit{buckets}, aunque en el estimador de pasos distribuidos lo llaman \textit{steps}.

\subsubsection{Distribuci\'on uniforme}

\begin{itemize}
\item \textbf{Operaci\'on por igualdad} \\

\begin{figure}[H]
	  \begin{center}
	    \includegraphics[scale=.80]{imagenes/parametroVariableC0Eq.png}
	    \caption{Error promedio de la Columna C0 de la tabla brindada por la materia} 
	    \label{fig:C0_variando_parametro}
	  \end{center}
\end{figure}

\quad En la figura \ref{fig:C0_variando_parametro} se ve como en donde se obtiene el mayor impacto al aumentar el par\'ametro, es tanto en el histograma clasico como en el estimador propio. Si bien en Steps el error disminuye, no lo hace tanto como en los dem\'as estimadores. Tambi\'en se puede observar como para ciertos valores del par\'ametro, el histograma cl\'asico realiza varios saltos. Esto posiblemente se deba a alg\'un caso borde o a la irregularidad de la distribuci\'on

\item \textbf{Operaci\'on por mayor} \\

\quad lala \\

\begin{figure}[H]
	  \begin{center}
	    \includegraphics[scale=.80]{imagenes/parametroVariableC0Greater.png}
	    \caption{Error promedio de la Columna C0 de la tabla brindada por la materia} 
	    \label{fig:C0_variando_parametro_greater}
	  \end{center}
\end{figure}

\quad lala \\

\end{itemize}

\subsubsection{Distribuci\'on normal}

\begin{itemize}
\item \textbf{Operaci\'on por igualdad} \\

\quad lala \\

\begin{figure}[H]
	  \begin{center}
	    \includegraphics[scale=.80]{imagenes/parametroVariableC2Eq.png}
	    \caption{Error promedio de la Columna C2 de la tabla brindada por la materia} 
	    \label{fig:C2_variando_paremetro}
	  \end{center}
\end{figure}

\quad lala \\

\item \textbf{Operaci\'on por mayor} \\

\quad lala \\

\begin{figure}[H]
	  \begin{center}
	    \includegraphics[scale=.80]{imagenes/parametroVariableC2Greater.png}
	    \caption{Error promedio de la Columna C2 de la tabla brindada por la materia} 
	    \label{fig:C2_variando_paremetro_greater}
	  \end{center}
\end{figure}

\quad lala \\

\end{itemize}