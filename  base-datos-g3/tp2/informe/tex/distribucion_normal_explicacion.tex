La distribuci\'on normal es una de las distribuciones que mas aparece en la vida real. A continuaci\'on se presentan 2 ejemplos de la misma.
	
	\subsubsection*{Altura de una persona}
	
		Es ampliamente conocido que los caracteres morfol\'ogicos de individuos, tales como la estatura, siguen el modelo normal en todo el mundo. A simple vista, se puede pensar como por lo general la altura de las personas suele estar entre 1-70 y 1.75 metros, o por lo menos en la mayor\'ia de los casos es as\'i. No suele haber muchas personas que midan menos de 1.50, y a la ves no hay tampoco demasiadas personas que superen los 2 metros. Haciendo un muestreo poblacional y realizando un histograma del mismo se puede visualizar esta intuici\'on.
	
\newpage	
		
\begin{figure}[H]
  \begin{center}
    %\includegraphics[scale=.41,angle=-90]{imgenes/normal_ejemplo1.png}
    \includegraphics[scale=.40]{imagenes/normal_ejemplo1.png}
    \caption{Histograma altura} 
    \label{fig:normal_ejemplo1}
  \end{center}
\end{figure}		
		
Se puede ver como para el muestreo realizado, la mayor\'a de las personas caen en la altura entre 1.70 y 1.80 metros, dando como resultado aproximadamente, una normal con media 1,75 y desv\'io standard 0.64.

	\subsubsection*{IQ de una persona}
	
		Otro ejemplo muy conocido es el de el coeficiente intelectual de las personas, tambi\'en llamado IQ. Seg\'un el siguiente ranking, vemos que una inteligencia normal deber\'ia esta entre los 90 y los 109 de coeficiente intelectual, por lo que es de esperar que la mayor parte de la poblaci\'on este en este promedio.
\newline

\begin{tabular}{| l | l |}
\hline
IQ Range & Clasificaci\'on \\
\hline
130 o mas & Muy Superior \\
\hline
120\--129	& Superior \\
\hline
110\--119	& Arriba del promedio \\
\hline
90\--109	& Promedio \\
\hline
80\--89	& Abajo del Promedio \\
\hline
70\--79	& Limite \\
\hline
69 o menos & Extremadamente bajo \\
\hline
\end{tabular}
\newline

\noindent 
Veamos un histograma sobre el muestreo del IQ de los individuos de una poblaci\'on. 

\newpage

\begin{figure}[H]
  \begin{center}
    %\includegraphics[scale=.41,angle=-90]{imgenes/normal_ejemplo1.png}
    \includegraphics[scale=.40]{imagenes/normal_ejemplo2.png}
    \caption{Histograma IQ} 
    \label{fig:normal_ejemplo2}
  \end{center}
\end{figure}	

En la figura \ref{fig:normal_ejemplo2}, podemos un ver como efectivamente la mayor\'ia de la poblaci\'on se centra en un IQ de 100, con una desv\'io al rededor de 20.