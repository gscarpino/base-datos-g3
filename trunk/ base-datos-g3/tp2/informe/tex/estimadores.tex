\quad En este trabajo veremos tres estimadores, que los explicaremos a continuaci\'on:

\begin{itemize}
\item \textbf{Histograma Cl\'asico: } este estimador divide el rango de los valores en varios subrangos llamados \textit{buckets}. Contabiliza los cada valor aumentando la cantidad del \textit{bucket} correspondiente. Se basa fuertemente en estimar la probabilidad de un valor v con la cantidad total del \textit{bucket} correspondiente a ese v. Cuanto mayores subrangos haya, m\'ayor va a ser la precisi\'on para determinar la frecuencia del valor y de ah\'i estimar su selectividad. \\

\item \textbf{Pasos Distribuidos: } estimador inventado por Piatetsky-Shapiro. Al igual que en los histogramas cl\'asicos, usa el concepto de \textit{bucket} pero en vez de usar un ancho (rango) fijo para cada bucket, la cantidad de elementos en un \textit{bucket} est\'a determinada por la altura del mismo. Depende de la cantidad de \textit{buckets} que se desea, dividiendo la cantidad total de elementos (tuplas) por esa cantidad se obtiene la altura.\\

\item \textbf{Estimador Propio: } Nos basamos en el histograma cl\'asico pero introducimos una peque\~na mejora (m\'as adelante se ver\'a en los an\'alisis de los testeos). Realizamos un histograma cl\'asico, luego detectamos una cantidad arbitraria de buckets con mayor cantidad de elementos y dividimos su rango en la mitad creando dos nuevos buckets. Cabe destacar que esto no significa que al dividir el rango en la mitad, se dividan la cantidad de elementos en la mitad. Para poder realizar esto, utilizamos un arreglo que mantiene informaci\'on espec\'ifica de cada \textit{bucket}, su rango. Notamos que la desventaja con respectoa  los otros dos estimadores es que en cada subdivisi\'on de un buckets necesitamos realizar una consulta a la base de datos para determinar cu\'antos elementos van en cada nuevo bucket. Para este trabajo, creamos el histograma cl\'asico con la mitad de buckets pasados como par\'ametro. Vamos dividiendo buckets que tengan el m\'aximo n\'umero de elementos hasta llegar a la cantidad de buckets indicado por par\'ametro.

\end{itemize}