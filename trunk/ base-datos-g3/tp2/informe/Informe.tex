\documentclass[10pt, a4paper]{article}

%\usepackage[paper=a4paper, left=1.5cm, right=1.5cm, bottom=1.5cm, top=3.5cm]{geometry}
\usepackage[paper=a4paper]{geometry}
\usepackage[latin1]{inputenc}
\usepackage[T1]{fontenc}
\usepackage[spanish]{babel}
\usepackage{indentfirst}
\usepackage{fancyhdr}
\usepackage{latexsym}
\usepackage{lastpage}
\usepackage[colorlinks=true, linkcolor=blue]{hyperref}
\usepackage{calc}
\usepackage{paralist}
\usepackage{caratula}
%\usepackage[plain,noline,linesnumberedhidden, noend]{algorithm2e}
%\usepackage[plain,noline,linesnumbered, noend]{algorithm2e}
\usepackage[plain,noline,noend]{algorithm2e}
\usepackage{graphicx}
\usepackage{caption}
\usepackage{subcaption}
\usepackage{amsmath}
\usepackage{float}
\usepackage{ulem}

\newcommand{\f}[1]{\text{#1}}
\newcommand{\call}[1] {\textsc{#1}}
\newcommand{\tab}[1]{\Indp \Indp #1 \Indm  \Indm}
\newcommand{\tabSpace}[1]{\Indp \Indp \; #1 \Indm \Indm \;}
\newcommand{\si}[2]{ {\bf si} (#1)\; \tab {#2} }
\newcommand{\sino}[1]{ {\bf sino}\; \tab {#1} }
\newcommand{\sinosi}[2]{ {\bf sino, si #1}\; \tab {#2} }
\newcommand{\mientras}[2] {  {\bf mientras} (#1) \; \tabSpace  {#2} }
\newcommand{\funcion}[3] { 	\KwSty{\call{#1}(#2)}\; \tabSpace  {#3} \KwSty{\call{Fin Funcion}} \newline }
\newcommand{\funcionConResultado}[4] {      \KwSty{\call{#1}(#2) $\rightarrow$ #3}\; \tabSpace  {#4} \KwSty{\call{Fin Funcion}} \newline }
\newcommand{\porcada}[3] { {\bf por cada } #1 {\bf en} #2 \; \tabSpace {#3} }
\newcommand{\para}[4] { {\bf para } #1 {\bf desde} #2 {\bf hasta} #3\; \{ \; \tab{#4}\}\; }
\newcommand{\paracada}[3] { {\bf para cada } #1 {\bf $\in$} #2 \; \tabSpace {#3} }
\newcommand{\devolver}[1] { {\bf devolver} #1\;}



\sloppy


\hypersetup{%
 % Para que el PDF se abra a pagina completa.
 pdfstartview= {FitH \hypercalcbp{\paperheight-\topmargin-1in-\headheight}},
 pdfauthor={C\'atedra de Algoritmos y Estructuras de Datos III - DC - UBA},
 pdfkeywords={Trabajo Pr\'actico 3},
 pdfsubject={}
}

\parskip=5pt % 10pt es el tama\~no de fuente

% Pongo en 0 la distancia extra entre itemes.
\let\olditemize\itemize
\def\itemize{\olditemize\itemsep=0pt}

% Acomodo fancyhdr.
\pagestyle{fancy}
\thispagestyle{fancy}
\addtolength{\headheight}{1pt}
\lhead{Base de Datos}
\rhead{TP2}
\cfoot{\thepage /\pageref{LastPage}}
\renewcommand{\footrulewidth}{0.4pt}

\author{Base de Datos, DC, UBA.}
\date{}
\title{}

\begin{document}

%Pagina de titulo e indice
\thispagestyle{empty}
\materia{Base de datos}
\submateria{TP2}
\titulo{}
\grupo{NN\_3}
\integrante{Sergio Gonz\'alez}{723/10}{sergiogonza90@gmail.com}
\integrante{Gino Scarpino}{392/08}{gino.scarpino@gmail.com}
\maketitle

\tableofcontents

\SetAlgoSkip{bigskip}
\NoCaptionOfAlgo
\DontPrintSemicolon
\SetAlFnt{\ttfamily}

\newpage

\section{Introducci\'on}

	\quad Es fundamental para cualquier motor de bases de datos, poseer un planificador para resolver consultas lo m\'as eficiente posible. Medir el costo de un m\'etodo de evaluaci\'on puede ser muy complejo, pero como se indica en el paper de Piatetsky-Shapiro, aproximadamente son las cantidad de operaciones de entrada y salida que el motor realiza. Por eso mismo, se trata de minimizar estas operaciones.

Una de las formas de poder minimizar estas operaciones, es el poder conocer aproximadamente cual puede ser la distribucion de un set de datos, y de esta forma poder estimar cuantas tuplas se obtendar por el echo de aplicar un filtro (WHERE) u otro. El echo de poder minimizar las tuplas resultantes que se obtendran en una consulta, puede hacer que al momento de materializar la misma (por ejemplo en caso de tener que hacer un JOIN posterior) haga que las bajadas a disco de las tuplas se minimize.

Si bien computar la distribucion exacta de un set de datos puede ser muy costoso, existen metodos con el cual se puede aproximar dichas distribuciones y de esa forma poder decidir cual es el mejor camino a seguir al momento de tener que resolver una consulta.




\section{Estimadores }

*******************\\
Aca no se si habria que explicar cada estimador.. tecnicamente no lo pide el enunciado.. asi q esto se podria obviar :P, si hay q explicar mas adelante el estimadir propio\\
*******************

\section{An\'alisis de m\'etodos}

	\subsection{Distribuciones utilizadas}
	
		\subsubsection{Distribucion normal (Ejemplos)}	
		
			La distribuci\'on normal es una de las distribuciones que mas aparece en la vida real. A continuaci\'on se presentan 2 ejemplos de la misma.
	
	\subsubsection*{Altura de una persona}
	
		Es ampliamente conocido que los caracteres morfol\'ogicos de individuos, tales como la estatura, siguen el modelo normal en todo el mundo. A simple vista, se puede pensar como por lo general la altura de las personas suele estar entre 1-70 y 1.75 metros, o por lo menos en la mayor\'ia de los casos es as\'i. No suele haber muchas personas que midan menos de 1.50, y a la ves no hay tampoco demasiadas personas que superen los 2 metros. Haciendo un muestreo poblacional y realizando un histograma del mismo se puede visualizar esta intuici\'on.
	
\newpage	
		
\begin{figure}[H]
  \begin{center}
    %\includegraphics[scale=.41,angle=-90]{imgenes/normal_ejemplo1.png}
    \includegraphics[scale=.40]{imagenes/normal_ejemplo1.png}
    \caption{Histograma altura} 
    \label{fig:normal_ejemplo1}
  \end{center}
\end{figure}		
		
Se puede ver como para el muestreo realizado, la mayor\'a de las personas caen en la altura entre 1.70 y 1.80 metros, dando como resultado aproximadamente, una normal con media 1,75 y desv\'io standard 0.64.

	\subsubsection*{IQ de una persona}
	
		Otro ejemplo muy conocido es el de el coeficiente intelectual de las personas, tambi\'en llamado IQ. Seg\'un el siguiente ranking, vemos que una inteligencia normal deber\'ia esta entre los 90 y los 109 de coeficiente intelectual, por lo que es de esperar que la mayor parte de la poblaci\'on este en este promedio.
\newline

\begin{tabular}{| l | l |}
\hline
IQ Range & Clasificaci\'on \\
\hline
130 o mas & Muy Superior \\
\hline
120\--129	& Superior \\
\hline
110\--119	& Arriba del promedio \\
\hline
90\--109	& Promedio \\
\hline
80\--89	& Abajo del Promedio \\
\hline
70\--79	& Limite \\
\hline
69 o menos & Extremadamente bajo \\
\hline
\end{tabular}
\newline

\noindent 
Veamos un histograma sobre el muestreo del IQ de los individuos de una poblaci\'on. 

\newpage

\begin{figure}[H]
  \begin{center}
    %\includegraphics[scale=.41,angle=-90]{imgenes/normal_ejemplo1.png}
    \includegraphics[scale=.40]{imagenes/normal_ejemplo2.png}
    \caption{Histograma IQ} 
    \label{fig:normal_ejemplo2}
  \end{center}
\end{figure}	

En la figura \ref{fig:normal_ejemplo2}, podemos un ver como efectivamente la mayor\'ia de la poblaci\'on se centra en un IQ de 100, con una desv\'io al rededor de 20.		
		
		\subsubsection{Distribucion uniforme (Ejemplos)}	
		
				La distribucion uniforme es una de las mas conocidas, y como la nomal, una de las mas presentes en la realidad.
	Esta distribuci\'on es muy simple. B\'asicamente plantea que tenemos un espacio muestral $S=\{m_1, m_2, ... ,m_n\}$, $(\forall m_i \in S$,  $P(m_i)=1/n)$, osea, todos los sucesos tiene la misma probabilidad de ocurrir. 
			
	El ejemplo mas cl\'asico para entender esta distribuci\'on, es el lanzamiento de un dado de 6 caras (no cargado), en donde $S = \{1, 2, 3, 4, 5, 6\}$ y cada elemento tiene la misma probabilidad de salir, osea $ \frac{1}{6} $.
			
\subsubsection*{Tiempo de espera de un colectivo}

	Otro ejemplo un poco m\'as interesante, es si tomamos un rango de tiempo, y medimos el tiempo de espera de un colectivo en ese rango. Si bien este ejemplo depender\'a mucho sobre que l\'inea de colectivo se haga el muestreo, se puede tomar un rango en particular en el cual sabemos que el tiempo de espera no ser\'a mayor o menor a eso. A continuaci\'on, se presenta un histograma sobre un \textit{dataset} sobre los tiempos de cierto colectivo en el rango de 5 minutos a 30 minutos.
		
	\begin{figure}[H]
	  \begin{center}
	    %\includegraphics[scale=.41,angle=-90]{imgenes/normal_ejemplo1.png}
	    \includegraphics[scale=.40]{imagenes/uniforme_ejemplo.png}
	    \caption{Histograma tiempos de espera de colectivo} 
	    \label{fig:normal_ejemplo1}
	  \end{center}
	\end{figure}

	\subsection{An\'alisis de los estimadores}
			
		\textbf{Pregunta}: Dados los m\'etodos ``Classic Histograms'' y ``Distribution Steps''. Cu\'al esperar\'ia que
tenga mejor performance en la estimaci\'on para un par\'ametro f\'ijo (en caso del primer
m\'etodo, el par\'ametro es la cantidad de bins y, en el segundo, la cantidad de steps?
Justif\'icar.			

		Al ser ``Distribution Steps'' un histograma que no tiene el mismo ancho, sino que tiene misma altura en todos los \textit{bins}, claramente es esperable que para un parametro fijo, y un set de datos fijo, ``Distribution Steps'' tenga mejor performance.
		
		Para comprobar esto, utilizaremos el set de datos provisto para la materia y realizaremos graficos comparando ambos metodos. 
		
		\subsubsection*{Caso 1}
		
		\begin{tabular}{| l | l |}
		\hline
		Parametro & 10 \\
		\hline
		Columna & C2 \\
		\hline
		Distribucion & Normal \\
		\hline
		\end{tabular}
			
				
		
		
\end{document}