\begin{itemize}
\item El atributo edad de cada legislador que sea diputado es de por lo menos 25.
\item El atributo edad  de cada legislador en la c�mara de senadores es de al menos 30.
\item La fecha de inicio de cargo de ser menor a la fecha de fin de cargo.
\item No puede haber un legislador que sea legislador y diputado en periodos de
 inicio de periodo y fin periodo solapados.

\item En todas las relaciones que tengan fechas como atributos las fechas de inicio
 deben ser anteriores a las de fin.

\item La cantidad de comisiones que la c�mara de diputados es de 45.

\item La fecha de una inicio y fin de una sesi�n puede estar entre el 1 de marzo al 30 de noviembre de un mismo a�o.

\item La suma de los votos totales de un proyecto de ley aprobado es igual a la suma de todos los legisladores que son diputados y todos los legisladores que son senadores, que estuvieron presentes en las sesiones.

\item Todos los legisladores que componen la c�mara de diputados, son diputados y todos los que componen la c�mara de senadores son senadores.

\item La cantidad de senadores en la c�mara de senadores es de 3* cantidad de provincias.

\item La cantidad de legisladores que asiste a cada sesi�n es menor o igual a la cantidad de legisladores que componen la c�mara de diputados o la cantidad de legisladores que componen la c�mara senadores.

\item Si un legislador  �l� tiene mas de 15 ausencias hasta determinada fecha, entonces no puede haber una tupla de la forma (l,s) en la relaci�n asiste donde �s� es una sesi�n, con fecha dentro del periodo de actividad del legislador.

\item Los atributos fecha de inicio y fecha fin de participaci�n en comisiones, presidencia de bloques o bien presidencia de comisiones deben estar incluidas en el periodo de de inicio y fin que tiene como legislador.

\item Si un diputado d, preside una comisi�n c, en una fecha de inicio i, y fin d. Entonces no existen dos tuplas t1 y t2 en la relaci�n �preside� de la forma(d,c) con fechas de inicio y fin solapadas. Es decir un diputado solo puede presidir una comisi�n a la vez.


\end{itemize}





















