En esta secci\'on presentamos algunas presunciones que hicimos producto de algunas omisiones en el enunciado o simplemente cosas que hacen al sentido com\'un de la realidad que se pretend\'ia modelar.

\begin{enumerate}
	\item Asumimos que una Provincia podr\'ia tener menos de 33000 habitantes.
	\item Asumimos que una comisi\'on tiene un \'unico informante.
	\item Siempre existe un Vicepresidente Nacional.
	\item Un bloque representa siempre al mismo partido pol\'itico.
	\item No se mide los bienes del tipo acciones por unidad sino en conjunto y con un valor entero.
	\item Los votos son por sesi\'on y los presentes votan todos juntos. Los que no votan, no se registran.
	\item El empleado que realiza el control de calidad, no es un legislador, ni tampoco lo fue o estuvo como legislador en la historia que guarda el modelo. Solo interesa su id de empleado y su nombre.
	\item Una sesi\'on extraordinaria que extiende el per\'iodo legislativo, siempre lo hace antes del 31 de Diciembre, por el tema de contabilizar las faltas en el a\~no.
\end{enumerate}
