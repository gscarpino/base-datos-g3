\subsubsection{Discusiones}

\quad Hubo varios temas puntuales por los cuales hubo que ponerse de acuerdo. Ya sea por gustos o porque tomar una desici\'on con respecto a algo afectaba otra parte de lo implementado.

\quad Adem\'as de decisiones sobre como implementar, tambi\'en hab\'ia que discutir y consensuar sobre interpretaciones del enunciado. Al principio de este trabajo hablamos sobre ciertas libertades que el problema dejaba.

\quad Los temas \textit{pol\'emicos} fueron:

\begin{itemize}
  \item C\'omo modelar el tema de los per\'iodos ya sea para los legisladores, el de las sesiones, los bienes, entre otras entidades.
  
  
  \item Caracter\'isticas propias de los bienes econ\'omicos, relevancia de las mismas.
  
  
  \item El tema del control de calidad de la c\'amara de senadores.
  
  
  \item El mecanismo de las votaciones de los legisladores sobre los proyectos de ley.
  
  
  \item Sobre las Sesiones y los tipos.
\end{itemize}

\quad El tema m\'as complejo fue, para nosotros, como modelar las votaciones en el DER.

\subsection{Conclusi\'on}

\quad Este trabajo pr\'actico nos permiti\'o comprender la importancia de modelar primero con un DER, luego con MR para, posteriormente, traducir esos modelos en los esquemas de la base de datos. Si uno no lo hiciera, ser\'ia altamente probable que, al hacerlo directamente en una base de datos, se encuentre con problemas de inconsistencia, contradicciones, o incompletitud.

\quad Pudimos ver, como decisiones en una parte del DER, pueden llegar a afectar otras partes, simplificar o volverlo m\'as complejo.