Cuando el equipo se encontr\'o en la tarea de desarrollar las consultas en SQL as\'i como el Schema, se tomaron algunas libertades con el fin de que se m\'as razonable y simple a la hora de hacer las consultas. Estas simplificaciones si bien no hacen grandes cambios a lo que es el modelo pertinente de la realidad, no coinciden en su totalidad con el diagrama DER. 

Algunas de las modificaciones efectuadas fueron:

En el DER se hace distinci\'on entre diputados y senadores, si bien a priori nos resulta muy representativo y modela la realidad que pretend\'iamos, nos encontramos con que en el momento de llevarlo a tablas cont\'abamos con una tabla para legisladores y otras para diputados y senadores con sus foreign keys respectivas. Esto nos pareci\'o un poco molesto y dif\'icil de comprender incluso tal vez a la hora de la correcci\'on por parte del docente, por ello decidimos tener una tabla legislador con el tipo, a pesar de conservar la idea original en el DER. 
Entendemos no obstante que el tipo NO ES un atributo del legislador ni mucho menos en el DER.

Otro cambio similar es el de las c\'amaras de diputados y senadores, dado que existe una sola c\'amara de cada uno y es evidente que en ellas habr\'a solo senadores o solo diputados, decidimos que la tabla tenga el tipo de c\'amara.